\documentclass[9pt,twocolumn,twoside,]{pnas-new}

% Use the lineno option to display guide line numbers if required.
% Note that the use of elements such as single-column equations
% may affect the guide line number alignment.


\usepackage[T1]{fontenc}
\usepackage[utf8]{inputenc}

% tightlist command for lists without linebreak
\providecommand{\tightlist}{%
  \setlength{\itemsep}{0pt}\setlength{\parskip}{0pt}}




\templatetype{pnasresearcharticle}  % Choose template

\title{Comment la diversité spécifique du benthos varie-t-elle dans les
rivières du Québec?}

\author[a]{Léanne Beauregard}
\author[a]{Élyse Castilloux}
\author[a]{Maude Roy}
\author[a]{Erick-Daniel Vasquez}

  \affil[a]{Univeristé de Sherbrooke, Département de Biologie, Street,
City, State, Zip}


% Please give the surname of the lead author for the running footer
\leadauthor{Anonymous}

% Please add here a significance statement to explain the relevance of your work
\significancestatement{}


\authorcontributions{}



\correspondingauthor{\textsuperscript{} }

% Keywords are not mandatory, but authors are strongly encouraged to provide them. If provided, please include two to five keywords, separated by the pipe symbol, e.g:
 \keywords{  Benthos |  Rivière  } 

\begin{abstract}
Dans le cadre du cours de Méthodes en écologie computationnelle (BIO500)
nous avons analysé les données de 43 rivières, récoltées entre 2016 et
2021, sur la biodiversité benthique sur les rivières du Québec. Plus
précisément, nous avons cherché à comprendre comment la diversité
spécifique du benthos varie en fonction de la vitesse du courant, de la
profondeur de la rivière et de la température de l'eau. Contrairement à
ce que nous pensions, les résultats démontrent que seule la température
de l'eau affecte significativement la richesse spécifique du benthos,
mais que ce facteur n'explique qu'une infime partie de sa variance
spatiale.
\end{abstract}

\dates{This manuscript was compiled on \today}
\doi{\url{www.pnas.org/cgi/doi/10.1073/pnas.XXXXXXXXXX}}

\begin{document}

% Optional adjustment to line up main text (after abstract) of first page with line numbers, when using both lineno and twocolumn options.
% You should only change this length when you've finalised the article contents.
\verticaladjustment{-2pt}



\maketitle
\thispagestyle{firststyle}
\ifthenelse{\boolean{shortarticle}}{\ifthenelse{\boolean{singlecolumn}}{\abscontentformatted}{\abscontent}}{}

% If your first paragraph (i.e. with the \dropcap) contains a list environment (quote, quotation, theorem, definition, enumerate, itemize...), the line after the list may have some extra indentation. If this is the case, add \parshape=0 to the end of the list environment.

\acknow{}

\section{Introduction}\label{introduction}

Les cours d'eau regorgent de petits organismes qu'on ne perçoit pas
toujours au premier abord, tel que les organismes benthiques. Le benthos
représente l'ensemble des macro-invertébrés qui vivent dans le fond des
cours d'eau et des lacs, comme des larves d'insectes, des crustacés et
des vers. Il est important de faire un suivi des organismes benthiques,
car ils peuvent être utilisés comme bio-indicateur, afin d'évaluer la
santé du cours d'eau et qu'ils servent de nourriture pour un grand
nombre d'animaux (\textbf{abrysiewicz\_characterisation\_2022?},
\textbf{melccfp\_macroinvertebres\_nodate?}). Cette présente recherche
cherche à caractériser comment la richesse spécifique varie d'une
rivière à l'autre au Québec, à l'aide de trois questions principales,
soit « comment la richesse spécifique du benthos varie en fonction de la
vitesse du courant », « comment la richesse spécifique du benthos varie
en fonction de la température de l'eau » et « comment la richesse
spécifique du benthos varie en fonction de la profondeur de la rivière
». Nous pensons que ces trois facteurs devraient influencer la richesse
spécifique du benthos, de manière positive ou négative.

\section{Méthode}\label{muxe9thode}

Nous avons premièrement rassemblé et nettoyé les données récoltées par
les équipes sur le terrain, pour créer une base de données commune où il
sera possible d'ajouter les nouvelles données terrain prises à chaque
année par la suite. Plus précisément, les données qui n'étaient pas
présentes pour la majorité des rivières ou qui étaient doublées ont été
éliminées et le format des données a également été homogénéisé. Ensuite,
diverses requêtes ont été utilisées afin de déterminer comment les
caractéristiques des rivières affectent la diversité spécifique du
benthos. Trois régressions linéaires ont été faites, la première étant
la richesse spécifique du benthos en fonction de la vitesse du courant,
la deuxième la richesse spécifique du benthos en fonction de la
profondeur de la rivière et la troisième la richesse spécifique du
benthos en fonction de la température de l'eau. Afin de mieux visualiser
ces résultats, trois différents tableaux ont été créés. Finalement, le
tout a été rassemblé sous le format d'article à l'aide de Rmarkdown.
L'ensemble de l'analyse et du nettoyage des données a été fait à l'aide
du logiciel R studio (2024.12.0+467). Les package ``dplyr'', ``readr'',
``DBI'', ``RSQLite'', ``janitor'', ``ggplot2'', ``rmarkdown'' et
``knitr'' ont été utilisés. La totalité du code est disponible sur
GitHub et rassemblée dans un pipeline target. Il faut noter que les
logiciels Chat GPT(4.0) et co-pilot ont été utilisés afin d'aider à
débugger les codes.

\section{Résultats}\label{ruxe9sultats}

La richesse spécifique du benthos varie de façon positive avec la
température de l'eau (pente = 0,5316, p=0,02946). Cependant, cette
relation n'explique qu'une infime partie de la variation du benthos dans
les différentes rivières, puisque le R2 est de 0,08632. Cette relation
est visible dans le \#\#\textbackslash label\{fig:plot1\}.

FIGURE 1

Pour ce qui est des autres régressions, celle de la richesse spécifique
en fonction de la rivière (p=0,293) et la vitesse du courant (p=0,532)
ont respectivement des pentes de -0,2350 et 0,4067, mais ces relations
ne sont pas significatives et ne peuvent donc pas être utilisées pour
expliquer la variation spatiale du benthos. La relation entre la
richesse spécifique et la vitesse du courant est illustrée dans le
deuxième graphique \#\#\textbackslash label\{fig:plot2\} , où la vitesse
a été séparée en différentes catégories, afin de mieux illustrer comment
elle varie. Puisque cette relation n'est pas significative, peu de
variation est observée entre les différentes catégories, avec une
diversité maximale lorsqu'il y a entre 0,4 et 0,6m de profondeur. Le
\#\#\textbackslash label\{fig:plot3\} quant à lui, représente comment la
diversité spécifique est distribuée selon différentes classes de vitesse
de courant : plus le pic d'une courbe est vers la droite et plus il est
haut, plus la chance de retrouver une forte diversité benthique dans
cette catégorie de vitesse est importante.

FIGURE 2 ET 3

\section{Discussion}\label{discussion}

Selon nos résultats, la profondeur des rivières n'est pas un élément qui
permet d'expliquer la différence dans la distribution spatiale du
benthos, car la corrélation entre les deux variables n'est pas
significative. Ce résultat nous étonne, car plusieurs études, dont une
réalisée au Québec, semblent avoir trouvé un lien entre la profondeur et
la différence en richesse spécifique
(\textbf{article?}\{brysiewicz\_characterisation\_2022,
\textbf{article?}\{tall\_effects\_2016). La profondeur étant liée à
d'autres paramètres comme la luminosité, la température et la
disponibilité en oxygène, un de ces facteurs pourrait peut-être mieux
expliquer la distribution spatiale du benthos. Tel que mentionné dans
les résultats, la vitesse du courant n'est pas significative non plus.
Encore une fois, cela est contraire à ce que l'on s'attendait, puisque
(\textbf{article?})\{barmuta\_interaction\_1990 et
(\textbf{article?})\{brysiewicz\_characterisation\_2022 ont trouvé une
relation significative entre ces variables. Ensuite, tel que nous
l'avions prédit, la température de l'eau a un impact sur la richesse
spécifique du benthos. Cependant, alors que dans les recherches
précédentes la température était un facteur important expliquant
l'évolution spatiale de la diversité benthique, selon notre analyse,
elle n'explique qu'une très petite partie de cette dernière.
(\textbf{article?})\{hawkins\_channel\_1997 et
(\textbf{article?})\{bonacina\_effects\_2023 suggèrent que certains
taxons ne peuvent tolérer des températures trop extrêmes : une
régression linéaire simple ne représente peut-être pas la relation entre
la température de l'eau et la richesse spécifique de la bonne manière,
ce qui pourrait expliquer pourquoi elle explique peu la dispersion
spatiale. Ainsi, dans une prochaine analyse, il serait pertinent de
tenter d'expliquer la relation entre ces deux variables à l'aide d'une
autre formule, par exemple la formule quadratique, afin de prendre en
compte la possibilité que l'eau diminue la diversité benthique
lorsqu'elle est trop froide et trop chaude. De manière semblable, les
résultats non-significatifs sont peut-être expliqués par le fait que la
distribution spatiale du benthos a été étudiée en un seul bloc. Il est
possible que certains taxons soient plus affectés par un paramètre que
d'autres, ou qu'ils soient affectés de manière contraire, ce qui aurait
pu nuire à établir une corrélation significative entre la vitesse du
courant ou la profondeur de la rivière et la richesse spécifique.
Finalement, il pourrait être intéressant faire des analyses multivariées
afin de tenter d'avoir des résultats plus significatifs. Soit en prenant
en compte plusieurs des variables étudiées dans cette présente analyse,
soit en prenant en compte d'autres facteurs qui n'étaient pas présents
dans les données récoltées, tel que le niveau d'oxygénation de l'eau ou
sa concentration en nitrogène, qui sont important pour expliquer la
diversité du benthos
(\textbf{article?})\{brysiewicz\_characterisation\_2022.

\section*{References}\label{references}
\addcontentsline{toc}{section}{References}

\bibliography{reference bio500.bib}

\showmatmethods
\showacknow
\pnasbreak



% Bibliography
% \bibliography{pnas-sample}

\end{document}
